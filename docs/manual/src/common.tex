\begin{titlepage}
  \begin{center}

  {\Huge UART\_PMOD1553}

  \vspace{25mm}

  \includegraphics[width=0.90\textwidth,height=\textheight,keepaspectratio]{img/AFRL.png}

  \vspace{25mm}

  \today

  \vspace{15mm}

  {\Large Jay Convertino}

  \end{center}
\end{titlepage}

\tableofcontents

\newpage

\section{Usage}

\subsection{Introduction}

\par
UART pmod1553 is a fusesoc 1553 project that has all input/output data through the UART. The 1553 is for avaiation bus connections.
The UART is a simple interface for sending and receiving data, and does not emulate a bus endpoint or controller. Software that
reads and writes from the UART is responsible for that response.

\subsection{Interface}
\par
For xilinx fifo and uart the format is the same.
Data is received in the following ASCII string format:
\begin{lstlisting}[language=bash]
  $ CMDS;D1;P1;I0;Hx5555\r
  $ DATA;D0;P1;I0;HxAAAA\r
\end{lstlisting}

\par
The fields are seperated by ; .

\begin{itemize}
\item The first is the sync type, Command/Status = CMDS, Data = DATA
\item The second is if there is a 4us delay, 1 = delay over 4us, 0 = no delay or less then 4us.
\item The third is parity, 1 = parity good, 0 = parity bad
\item The fourth is invert, 1 = core is inverting data, 0 = core is not inverting data
\item The fifth is the data in hex format, Hx???? where ? = 4 characters representing the data (16 bits in total).
\item The carrige return is the string terminator. This works well with serial consoles with local newline addition enabled.
\end{itemize}

\par
Data is sent in the following ASCII string format:
\begin{lstlisting}[language=bash]
  $ CMDS;D1;P1;I0;Hx5555\r
  $ DATA;D0;P1;I0;HxAAAA\r
\end{lstlisting}

\par
The fields are seperated by ; .

\begin{itemize}
\item The first is the sync type, Command/Status = CMDS, Data = DATA
\item The second is to enable a 4us delay, 1 = delay of at least 4us, 0 = no delay or less then 4us.
\item The third is parity, 1 = parity odd (default), 0 = parity even
\item The fourth is invert, 1 = invert data, 0 = don't invert data
\item The fifth is the data in hex format, Hx???? where ? = 4 characters representing the data (16 bits in total).
\item The carrige return is the string terminator.
\end{itemize}

\subsection{Dependencies}

\par
The following are the dependencies of the cores.

\begin{itemize}
  \item fusesoc 2.X
  \item iverilog (simulation)
  \item cocotb (simulation)
\end{itemize}

\subsubsection{fusesoc\_info Depenecies}
\begin{itemize}
\item nexys-a7-100t
	\begin{itemize}
	\item AFRL:utility:digilent\_nexys-a7-100t\_board\_base:1.0.0
	\end{itemize}
\item cmod-s7-25
	\begin{itemize}
	\item AFRL:utility:digilent\_cmod-s7-25\_board\_base:1.0.0
	\end{itemize}
\item arty-a7-35
	\begin{itemize}
	\item AFRL:utility:digilent\_arty-a7-35\_board\_base:1.0.0
	\end{itemize}
\item dep
	\begin{itemize}
	\item AFRL:project\_ip:uart\_1553\_core:1.0.0
	\item AFRL:utility:helper:1.0.0
	\item AFRL:utility:vivado\_board\_support\_packages
	\end{itemize}
\end{itemize}


\section{Architecture}
\par
The project contains wrappers for the top level projects, and a common uart 1553 module for all.

\begin{itemize}
  \item \textbf{system\_wrapper} Contains the top level project module and contains uart\_1553\_core and any other logic needed.
  \item \textbf{uart\_1555\_core} Contains the PMOD1553, UART, and string generation interfaces.
\end{itemize}

\par

Please see \ref{Module Documentation} for more information per target.

\section{Building}

\par
The all UART pmod1553 is written in Verilog 2001. They should synthesize in any modern FPGA software. The core comes as a fusesoc packaged core and can be
included in any other core. Be sure to make sure you have meet the dependencies listed in the previous section.

\subsection{fusesoc}
\par
Fusesoc is a system for building FPGA software without relying on the internal project management of the tool. Avoiding vendor lock in to Vivado or Quartus.
These cores, when included in a project, can be easily integrated and targets created based upon the end developer needs. The core by itself is not a part of
a system and should be integrated into a fusesoc based system. Simulations are setup to use fusesoc and are a part of its targets.

\subsection{Source Files}

\subsubsection{fusesoc\_info File List}
\begin{itemize}
\item src\_xilinx
	\begin{itemize}
	\item {'common/xilinx/system\_gen.tcl': {'file\_type': 'tclSource'}}
	\end{itemize}
\item nexys-a7-100t
	\begin{itemize}
	\item {'nexys-a7-100t/system\_constr.xdc': {'file\_type': 'xdc'}}
	\item {'nexys-a7-100t/system\_wrapper.v': {'file\_type': 'verilogSource'}}
	\end{itemize}
\item cmod-s7-25
	\begin{itemize}
	\item {'cmod-s7-25/system\_constr.xdc': {'file\_type': 'xdc'}}
	\item {'cmod-s7-25/system\_wrapper.v': {'file\_type': 'verilogSource'}}
	\end{itemize}
\item arty-a7-35
	\begin{itemize}
	\item {'arty-a7-35/system\_constr.xdc': {'file\_type': 'xdc'}}
	\item {'arty-a7-35/system\_wrapper.v': {'file\_type': 'verilogSource'}}
	\end{itemize}
\end{itemize}


\subsection{Targets} \label{targets}

\subsubsection{fusesoc\_info Targets}
\begin{itemize}
\item default
	\begin{itemize}
	\item[$\space$] Info: Default files, invalid gen target.
	\end{itemize}
\item nexys-a7-100t
	\begin{itemize}
	\item[$\space$] Info: Nexyx a7 100t dev board target.
	\end{itemize}
\item cmod-s7-25
	\begin{itemize}
	\item[$\space$] Info: cmod s7 25 dev board target.
	\end{itemize}
\item arty-a7-35
	\begin{itemize}
	\item[$\space$] Info: arty a7 35 dev board target.
	\end{itemize}
\end{itemize}


\subsection{Directory Guide}

\par
Below highlights important folders from the root of the directory.

\begin{enumerate}
  \item \textbf{docs} Contains all documentation related to this project.
    \begin{itemize}
      \item \textbf{manual} Contains user manual and github page that are generated from the latex sources.
    \end{itemize}
  \item \textbf{arty-a7-35} Contains source files for arty-a7-35
  \item \textbf{common} Contains source file wrapper and helper cores for uart PMOD1553 core
  \item \textbf{cmod-s7-25} Contains source files for cmod-s7-25
  \item \textbf{nexys-a7-100t} Contains source files for nexys-a7-100t
\end{enumerate}

\newpage

\section{Simulation}
\par
There is no simulation at the moment. This is dues to the AD9361 and ARM subsystems. Maybe a future addition with Vexriscv?

\newpage

\section{Module Documentation} \label{Module Documentation}

\par
There project has multiple modules. The targets are the top system wrappers.

\begin{itemize}
\item \textbf{arty-a7-35 system wrapper}
\item \textbf{cmod-s7-25 system wrapper}
\item \textbf{nexys-a7-100t system wrapper}
\item \textbf{uart 1553 core wrapper}
\item \textbf{axis 1553 string decoder}
\item \textbf{axis 1553 string encoder}
\item \textbf{axis char to string converter}
\end{itemize}
The next sections document the modules.

